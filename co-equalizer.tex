
%%% Local Variables: 
%%% mode: latex
%%% TeX-master: t
%%% End: 


\documentclass[a4paper]{amsart}            %for bookmarks enable option [liststotoc]%



%-------packages-------------------------%


\usepackage{amsmath}
\usepackage{amsthm}        % Does theorem stuff
\usepackage{amssymb} 
\usepackage{xypic}  				%for strange reason I need this to make the two cell diagrams				
\usepackage[2cell]{xy} 			%for commutative diagrams%

\usepackage{color,hyperref}
\definecolor{darkblue}{rgb}{0.0,0.0,0.3}                      %pdf book marks the way I like%
\hypersetup{pdftex=true, colorlinks=true, breaklinks=true, linkcolor=darkblue, menucolor=darkblue, pagecolor=darkblue, urlcolor=darkblue}



%-----------style------------------------%

\addtolength{\parskip}{\baselineskip} %Absätze im Text werden auch tatsächlich zu Absätzen%
\parindent 0pt

%----------new--commands-----------------%

\newcommand{\C}{\mathcal{C}}
\renewcommand{\O}{\mathcal{O}}
\newcommand{\D}{\mathcal{D}}
\newcommand{\F}{\mathcal{F}}
\newcommand{\G}{\mathcal{G}}
\newcommand{\ve}{\varepsilon}
\newcommand{\Mor}{\mathrm{Mor}}
\newcommand{\id}{\operatorname{id}}
\newcommand{\Hom}{\operatorname{Hom}}

%this command was proposed by Stefan_K%
\newcommand*{\leftterm}{\left\{\substack{\normalsize \text{natural transformations}
  \\h^A \rightarrow F} \right\}}
\newlength{\leftside}
\settowidth{\leftside}{$\leftterm$}

% the following code will uplift the \maketitle title. In standard it is way too low.
\makeatletter % wegen @ in den Befehsnamen
\renewcommand*\@maketitle{%
  \normalfont\normalsize
  \@adminfootnotes
  \@mkboth{\@nx\shortauthors}{\@nx\shorttitle}%
% (SCHW) auskommentiert:  \global\topskip42\p@\relax % 5.5pc   "   "   "     "     "
  \@settitle
  \ifx\@empty\authors \else \@setauthors \fi
  \ifx\@empty\@dedicatory
  \else
    \baselineskip18\p@
    \vtop{\centering{\footnotesize\itshape\@dedicatory\@@par}%
      \global\dimen@i\prevdepth}\prevdepth\dimen@i
  \fi
  \@setabstract
  \normalsize
  \if@titlepage
    \newpage
  \else
    \dimen@34\p@ \advance\dimen@-\baselineskip
    \vskip\dimen@\relax
  \fi
} % end \@maketitle
\makeatother

%--------new--enviroments----------------%

\renewcommand{\theequation}{\thesection.\arabic{equation}}         %%                             %I somehow need this for the warning symbol to work 
 \makeatletter                                                      %%                             properly, I don't really know why though%
    \@addtoreset{equation}{section} % Make the equation counter reset each section
    \@addtoreset{footnote}{section} % Make the footnote counter reset each section
                                                                    %%
 \newenvironment{warning}[1][]{%                                    %%
    \begin{trivlist} \item[] \noindent%                             %%
    \begingroup\hangindent=2pc\hangafter=-2                         
    \clubpenalty=10000%                                             										
    \hbox to0pt{\hskip-\hangindent\manfntsymbol{127}\hfill}\ignorespaces%
    \refstepcounter{equation}\textbf{Warning~\theequation}%         
    \@ifnotempty{#1}{\the\thm@notefont \ (#1)}\textbf{.}            
    \let\p@@r=\par \def\p@r{\p@@r \hangindent=0pc} \let\par=\p@r}%  
    {\hspace*{\fill}$\lrcorner$\endgraf\endgroup\end{trivlist}}  
    
    
%\newenvironment{beweis}{\par\begingroup%
%\settowidth{\leftskip}{\emph{Proof.~}}%																											remove to use \begin{beweis}
%\noindent\llap{\emph{Proof.~}}}{\hfill$\Box$\par\endgroup}     
    
  

\newenvironment{tmo}[1]{%
  \trivlist
  \leftskip=0.15cm
  \item[\hskip\labelsep
        \bfseries
   #1\@{.}]\mbox{ }\par\nobreak
   \vskip -0.5em\nobreak% Absatzabstand nachträglich entfernen.
   \noindent
  \leftskip=0.35cm
  \rightskip=0.35cm
  \itshape\ignorespaces
}{%
\endtrivlist}

\newcounter{tm}
% Falls tm abhängig von \section nummeriert werden soll:
%\numberwithin{tm}{section}% anderenfalls auskommentieren!!!
\newenvironment{tm}[1]{% wie tmo aber mit Nummer
  \refstepcounter{tm}%
  \tmo{#1~\thetm}%
}{%
  \endtmo
}
     
\makeatletter

\renewenvironment{proof}[1][\proofname]{\par
  \pushQED{\qed}%
  \normalfont \topsep6\p@\@plus6\p@\relax
  \trivlist
  \leftskip=0.6cm
  
  \item[\hskip\labelsep
        \itshape
    #1\@addpunct{.}]\mbox{ }\par\noindent%
  \leftskip=1cm
  \rightskip=1cm    
}{%
  \popQED\endtrivlist\@endpefalse
}
\makeatother 

\makeatother
\theoremstyle{plain}                                               %%
 \newtheorem{theorem}[equation]{Theorem}                            %%
 \newtheorem*{claim}{Claim}                                         %%
 \newtheorem*{lemma*}{Lemma}
 \newtheorem*{proposition*}{Proposition}                                       %%
 \newtheorem*{theorem*}{Theorem}                                    %%
 \newtheorem{lemma}[equation]{Lemma}                                %%
 \newtheorem{corollary}[equation]{Corollary}                        %%
 \newtheorem{proposition}[equation]{Proposition}                    %%

%--produce-the-warning--symbol-----------%
 \DeclareFontFamily{U}{manual}{}                                
 \DeclareFontShape{U}{manual}{m}{n}{ <->  manfnt }{}            
 \newcommand{\manfntsymbol}[1]{%                                
    {\fontencoding{U}\fontfamily{manual}\selectfont\symbol{#1}}}

\makeatother


\begin{document}

%------------header----------------------%

\title{Equalizer and Coequalizer}
%\author{Felix Hoffmann}% %for now I decided against it%

\begin{abstract}In this short article basic properties of equalizers and their dual concept of coequalizers are being explored. In particular, equalizers and coequalizers in \textbf{Set} are characterized. In the last part it is shown that subspace toplogy constructions correspond to equalizers in \textbf{Top}, whereas, dually, quotient topology constructions correspond to coequalizers.
\end{abstract}

\maketitle
\pagestyle{empty} %no pagenumbers and titles%
\thispagestyle{empty}



%-------end header-----------------------%

Equalizers (Coequalizers) are limits (colimits) over diagrams

\begin{align*}
\xymatrix@=+1.25cm{ {\bullet} \ar@<-1ex>[r] \ar@<1ex>[r] & {\bullet}}
\end{align*}

with two objects and two parallel arrows.

An equalizer of $\xymatrix{X \ar@<0.66ex>[r]^{f} \ar@<-0.66ex>[r]_{g} & Y}$ is therefore a pair $(A,\iota)$, consisting of an object $A$ and a morphism $A \xrightarrow{\iota} X$, such that

\begin{enumerate}

\item $f \circ \iota = g \circ \iota$
\bigskip
\item and for every other object $T$ with a morphism $T \xrightarrow{t} X$, such that $f \circ \iota = g \circ \iota$, there exists a unique morphism $T \rightarrow A$ such that 

\begin{align*} \xymatrix@+=1.25cm{A \ar[r]^{\iota} & X \ar@<0.66ex>[r]^f \ar@<-0.66ex>[r]_g & Y \\
																										 T \ar@{.>}[u] \ar[ur]_t} \end{align*}
																										 
																										 commutes.
																										 																										 
\end{enumerate}	

\textbf{Note} that a morphism $A \rightarrow Y$ need not be specified since it is already uniquely determined by the demand of commutativity of the following cone:

\begin{align*} \xymatrix@=0.6cm{&A \ar@/_/[ddl] \ar@{.>}@/^/[ddr] \\ & & \\X \ar@<0.66ex>[rr]^f \ar@<-0.66ex>[rr]_g	 & & Y}
\end{align*}

In the same way the requirement, that $f \circ \iota = g \circ \iota$, also follows directly from the limit property of $(A, \iota)$.

A coequalizer of $\xymatrix{X \ar@<0.66ex>[r]^{f} \ar@<-0.66ex>[r]_{g} & Y}$ is then a pair $(B,\pi)$, consisting of an object $B$ and a morphism $Y \xrightarrow{\pi} B$, such that

\begin{enumerate}

\item $\pi \circ f = \pi \circ g$
\bigskip
\item and for every other object $T$ with a morphism $Y \xrightarrow{t} T$, such that $t \circ f = t \circ g$, there exists a unique morphism $B \rightarrow T$ such that 

\begin{align*} \xymatrix@+=1.25cm{X \ar@<0.66ex>[r]^f \ar@<-0.66ex>[r]_g & Y \ar[dr]_t \ar[r]^{\pi} & B \ar@{.>}[d]\\
																								&	&	 T } \end{align*}
																										 
																										 commutes.
																										 																										 
\end{enumerate}																							

Per definition the notion of equalizer is \emph{dual} to the notion of coequalizer. From abstract nonsense we also know that any two equalizers resp. coequalizers are uniquely isomorphic.
\bigskip

\section*{\textbf{Equalizers and Coequalizers in Set}}

Given $\xymatrix{X \ar@<0.66ex>[r]^f \ar@<-0.66ex>[r]_g & Y}$ in \textbf{Set} we can easily form the equalizer of this diagram as $(A,\iota)$, where
\begin{equation*}
A=\left\{x \in X \mid f(x)=g(x)\right\}
\end{equation*}

and $\iota:A \hookrightarrow X$ is the natural inclusion map from $A$ into $X$.

Indeed, by definition $f \circ \iota = g \circ \iota$ holds, and given another set $T$, such that $f \circ t  = g \circ t$, we conclude that $t(x) \in A \subset X$ (since $f(t(x)) = g(t(x))$!) and have the unique map $\phi: T \rightarrow A, x \mapsto t(x)$ such that

\begin{align*}
\xymatrix@+=1.25cm{T \ar[d]_{\phi} \ar[rd]^t & & \\
									 A \ar@{^{(}->}[r]^{\iota} & X \ar@<0.66ex>[r]^f \ar@<-0.66ex>[r]_g & Y} 
\end{align*}

commutes.

\bigskip

Dually, we can form the coequalizer of $\xymatrix{X \ar@<0.66ex>[r]^f \ar@<-0.66ex>[r]_g & Y}$ as $(B, \pi)$, where $B = Y/{\sim}$ and $\pi$ is the canonical projection map 

\begin{equation*}
\xymatrix@=0.7cm{Y \ar[r]^{\pi} & Y/{\sim}}
\end{equation*}

Here $\sim$ is the equivalence relation that is generated by $f(x) = g(x)$ for all $x \in X$.

Of course $\pi \circ f = \pi \circ g$ holds, 

\begin{equation*}
\pi(f(x)) = [f(x)] = [g(x)] = \pi(g(x))
\end{equation*}

and given another object $T$, with map $Y \xrightarrow{t} T$, such that $t \circ f = t \circ g$, we define the map 

\begin{equation*}
\phi: Y/{\sim} \rightarrow T, [y] \mapsto t(y)
\end{equation*}

and conclude that $\phi$ is well-defined:

Given $y \sim z$, then, by definition, there is a chain $y=a_0,a_1,...,a_k=z$, such that for $i$ in ${1,...,k}$
	
\begin{align*}
&a_{i-1} = f(x) \mathrm{\ and} \ a_{i}=g(x), \mathrm{ \ or} \\
&a_{i-1} = g(x) \mathrm{\ and} \ a_{i}=f(x) \mathrm{\ for} \ x \mathrm{\ in} \ X
\end{align*}

Thus $t(y)=t(a_0)=t(a_1)=...=t(a_k)=t(z)$ and t is well-defined. Of course, $\phi$ is the only map such that 

\begin{align*} \xymatrix@+=1.25cm{X \ar@<0.66ex>[r]^f \ar@<-0.66ex>[r]_g & Y \ar[dr]_t \ar@{->>}[r]^{\pi} & Y/{\sim} \ar@{.>}[d]^{\phi}\\
																								&	&	 T } \end{align*}   	
																								
commutes.																								

\section*{\textbf{Equalizer morphisms are monomorphisms}}

Let $(A, \iota)$ be an equalizer of $\xymatrix{X \ar@<0.66ex>[r]^f \ar@<-0.66ex>[r]_g & Y}$, then $\iota: A \rightarrow X$ is a monomorphism.

\begin{proof}
Given two morphisms $h_1, h_2: Z \rightarrow A$, such that 

\begin{equation*}
\iota \circ h_1 = \iota \circ h_2
\end{equation*}

we need to show that $h_1 = h_2$.

This, however, follows directly from the universal property of $(A, \iota)$, 

\begin{align*}
\xymatrix@+=1.25cm{Z \ar@<-0.66ex>[d]_{h_1} \ar@<0.66ex>[d]^{h_2} \ar[rd]^{\iota \circ h_1 = \iota \circ h_2} & & \\
									 A \ar[r]^{\iota} & X \ar@<0.66ex>[r]^f \ar@<-0.66ex>[r]_g & Y}
\end{align*}

since 
\begin{equation*}
f \circ (\iota \circ h_1) = f \circ \iota \circ h_1 = g \circ \iota \circ h_1 = g \circ (\iota \circ h_1)
\end{equation*}

holds, there is only \emph{one} morphism $Z \rightarrow A$ makes the above diagram commute, it follows that $h_1 = h_2$.
\end{proof}

Dually, of course, \emph{coequalizer morphisms are epimorphisms}.

\bigskip

In \textbf{Set} the converse holds:

\begin{tm}{Proposition}
If $A \xrightarrow{\iota} X$ is a monomorphism, $(A, \iota)$ is an equalizer.
\end{tm}

\begin{proof}
To see this consider $X/{\sim}$, where $\sim$ is the induced equivalence relation on $X$ by $\iota(a_1)=\iota(a_2)$ for all $a_1,a_2 \in A$.

Assuming that $A \neq \emptyset$, denote $\ast$ as the equivalence class of any $a \in \operatorname{im} A$.

Then $(A, \iota)$ is an equalizer of $\xymatrix{X \ar@<0.66ex>[r]^-{\pi} \ar@<-0.66ex>[r]_-{\ast} &X/{\sim}}$, where $\pi$ is the canonical projection and $\ast$ is the \emph{constant} function that maps every element of X onto $\ast$ in $X/{\sim}$.

Indeed, $\pi \circ \iota = \ast \circ \iota$ holds, and given another object $T$ with morphism $T \xrightarrow{t} X$, such that $\pi \circ t = \ast \circ t$, we contend that $t(x) \in \operatorname{im} A$ for all $x$ in $X$.

For this assume $t(x) \notin \operatorname{im} A$, but then, by definition of $\sim$, $\pi(t(x)) = [t(x)] \neq \ast$, which contradicts the definition of $t$.

Thus we can define the map $\phi:T \rightarrow A, x \mapsto \iota^{-1}(t(x))$ and immediately verify that it satisfies the commutativity requirement. Also, $\phi$ is unique by the assumption that $\iota$ is a monomorphism.

If A is empty, $(A,\iota)$, where $\iota$ is in this case the empty function, is the equalizer of $\xymatrix{X \ar@<0.66ex>[r]^-{f} \ar@<-0.66ex>[r]_-{g} & X \times \{0,1\}}$, where $f(x)=(x,0)$ and $g(x)=(x,1)$.




\end{proof}

\bigskip 

There is a similar construction for coequalizers:

\medskip

\begin{tm}{Proposition}
If $Y \xrightarrow{\pi} B$ is an epimorphism in \textbf{Set}, $(B, \pi)$ is a coequalizer.
\end{tm}

\begin{proof}
We contend that $(B,\pi)$ is a coequalizer of $\xymatrix{Y \ar@<0.66ex>[r]^-{\id_Y} \ar@<-0.66ex>[r]_-{f} &Y}$, where $\id_Y$ is the identity map and $f$ is defined by $f(y)=\tau([y])$, where $\tau$ is any map 

\[
\tau: Y/{\sim} \to Y,
\]

 such that $[\tau([y])]=[y]$. 
\begin{align*}
\xymatrix@+=1cm{Y  \ar[rr]^{f} \ar[rd]_{\operatorname{can}} & & Y\\
								&	 Y/{\sim} \ar[ur]_{\tau}  &}
\end{align*}

Here, $\sim$ is the equivalence relation induced by $\pi$, 
\[y_1 \sim y_2 \Leftrightarrow \pi(y_1)=\pi(y_2).\]

The map $f$ therefore maps two elements $y_1,y_2$ in $Y$ to the same $f(y_1)=f(y_2)$, if and only if $\pi(y_1)=\pi(y_2)$ and we have the identity 
\[\pi(f(y))=\pi(y).\]

The assertion is now easily verfied. Firstly the equation above is exactly the demanded identity of $\pi \circ f = \pi \circ \id_Y$ and for the second property consider a set $T$ and map $t: Y \to T$ such that $t(f(y))=t(y)$. 

Since $\pi$ is surjective and by the requirement to commute with $\pi$ and $t$, the map $B \to T$ is already uniquely determined by $\pi(y) \mapsto t(y)$, and it is well-defined: Given $y_1,y_2$ in $Y$ with $\pi(y_1)=\pi(y_2)$, it follows that $f(y_1)=f(y_2)$ and thus 
\[t(y_1)=t(f(y_1))=t(f(y_2))=t(y_2).\]\end{proof}

\section*{\textbf{Equalizer and coequalier in Top}}


Equalizers in \textbf{Top} correspond to \emph{subspace topology} constructions, whereas coequalizers correspond to \emph{quotient topology} constructions.

First consider equalizers. We will show that to every subspace topology construction we can find a diagram that the subspace topology is an equalizer of. Then, conversely, we verify that every equalizer in \textbf{Top} is already a subspace topology construction.

We start with a topological space $(X, \O_X)$ and a subset $A$ of $X$. Then the pair $((A, \O_A), \iota)$, consisting of the topological space $(A, \O_A)$ and the map $\iota: A \rightarrow X$, where $\iota$ is the natural inclusion map of $A$ in $X$ and $\O_A=\{\iota^{-1}(U) \mid U \in \O:X \}$ is the subspace topology on A, is an \emph{equalizer} of $\xymatrix{(X,\O_X) \ar@<0.66ex>[r]^{\pi} \ar@<-0.66ex>[r]_{\ast} &(Y,\O_Y)}$.

Here $Y=X/{\sim},\ \pi$ and $\ast$ are constructed as in Proposition 1 and $\O_Y = \{U \subseteq Y \mid \pi^{-1} \in \O_X\}$ is the quotient topology on Y.

Indeed, from the construction in Proposition 1 we already know that $(A, \iota)$ is the set-theoretical equalizer of $\xymatrix{(X,\O_X) \ar@<0.66ex>[r]^{\pi} \ar@<-0.66ex>[r]_{\ast} &(Y,\O_Y)}$, so we just need to check the additional requirements in \textbf{Top}. The maps $\pi$ and $\ast$ are indeed continuous (and therefore proper morphisms in \textbf{Top}), since the canonical surjection $\pi$ is continuous by definition of $\O_Y$ and constant maps such as $\ast$ are always continuous.

Given another topological space $(T, \O_T)$ with a continuous map $T \xrightarrow{t} X$, such that $\pi \circ t = \ast \circ t$, we know from Proposition 1, that there is a unique map $\phi:T \rightarrow A$ such that     

\begin{align*} \xymatrix@+=1.25cm{(A,\O_A) \ar[r]^{\iota} & (X,\O_X) \ar@<0.66ex>[r]^{\pi} \ar@<-0.66ex>[r]_{\ast} & (Y,\O_Y) \\
																										 (T,\O_T) \ar[u]_{\phi} \ar[ur]_t} \end{align*}
																										 
commutes. But since $\iota \circ \phi = t$ is continuous, from the characteristic property of $\O_A$ it follows that $\phi:(T,\O_T) \rightarrow (A,\O_A)$ is continuous and thus a morphism in \textbf{Top}. 

\bigskip

For the converse direction, let $((A,\O_A), \iota)$ be an equalizer of a diagram \[\xymatrix{(X,\O_X) \ar@<0.66ex>[r]^{f} \ar@<-0.66ex>[r]_{g} &(Y,\O_Y)}\] in \textbf{Top}. We contend that the subspace topology construction $(\iota(A), \O_{\iota(A)})$, with the subset $\iota(A)$ in $X$, is already homeomorphic to $(A,\O_A)$. 

This is easily verified by using the universal property of $((A, \O_A), \iota)$: We show that $((\iota(A), \O_{\iota(A)}), j)$, where $j$ is the natural inclusion map, is itself an equalizer. Set theoretically, we see that $\iota': A \to \iota(A),\, a \mapsto \iota(a)$ is the unique bijection ($\iota'$ is injective) between $A$ and $\iota(A)$  that makes 

\begin{align*} \xymatrix@+=1.25cm{(\iota(A),\O_{\iota(A)}) \ar[r]^{j} & (X,\O_X) \ar@<0.66ex>[r]^{f} \ar@<-0.66ex>[r]_{g} & (Y,\O_Y) \\
																										 (A,\O_A) \ar[u]_{\iota'} \ar[ur]_\iota} \end{align*}

commute. From \[f(j(\iota'(x))=f(\iota(x))=g(\iota(x))=g(j(\iota'(x)))\] we conclude that $f \circ j = g \circ j$ holds. 

Then, given any other topological space $(T, \O_T)$ with continuous map $T \xrightarrow{t} X$, such that $f \circ t = g \circ t$ we get the unique continuous map $\phi: T \to A$ from the universal property of $(A, \O_A)$ and thus the unique map $\iota' \circ \phi: T \to \iota(A)$ such that 

\begin{align*} \xymatrix@+=1.25cm{(\iota(A),\O_{\iota(A)}) \ar[r]^{j} & (X,\O_X) \ar@<0.66ex>[r]^{f} \ar@<-0.66ex>[r]_{g} & (Y,\O_Y) \\
																										 (A,\O_A) \ar[u]_{\iota'} \ar[ur]_\iota \\(T,\mathcal{O}_T) \ar[u]^{\phi} \ar[uur]_t } \end{align*}
																										 
commutes. Since $j \circ (\iota' \circ \phi) = t$ is continuous, and $j$ is continuous by the definition of $\mathcal{O}_{\iota{A}}$, the map $\iota' \circ \phi$ is continuous by the characteristic property of the subspace topology. 

Thus $(\iota(A),\O_{\iota(A)}, j)$ is itself an equalizer of \[\xymatrix{(X,\O_X) \ar@<0.66ex>[r]^{f} \ar@<-0.66ex>[r]_{g} &(Y,\O_Y)}\] and hence have a homeomorphism between $(A,\O_A)$ and $(\iota(A), \O_{\iota(A)})$. The equalizer $(A,\O_A)$ may therefore be viewed as a subspace topology construction.

\bigskip

In the \emph{dual} case of coequalizers we proceed analoguously. First consider a quotient topology construction $(B,\O_{\operatorname{Quot(\pi)}})$ to a surjective map $\pi: Y \to B$ and a topology $(Y,\O_Y)$ on $Y$. 

From Proposition 2, we obtain a map $f$ such that $(B,\O_{\operatorname{Quot(\pi)}})$ is a coequalizer of the diagram \[\xymatrix{(Y,\O_Y) \ar@<0.66ex>[r]^-{\id_Y} \ar@<-0.66ex>[r]_-{f} &(Y,\O_Y)}\] in \textbf{Set}, and contend that this is also true in \textbf{Top}. Again, $\pi \circ f = \pi \circ \id_Y$ of course still holds and, as in the case of equalizers, the continuity of the unique map $B \rightarrow T$ follows from the characteristic property of the quotient topology. 

Conversely, given a coequalizer $((B,\O_B),\pi)$ of $\xymatrix{(X,\O_X) \ar@<0.66ex>[r]^{f} \ar@<-0.66ex>[r]_{g} &(Y,\O_Y)}$ we show that $(B,\O_{\operatorname{Quot(\pi)}})$ is a coequalizer of the same diagram. 

Here $\pi \circ f = \pi \circ g$ holds by definition for $(B,\O_{\operatorname{Quot(\pi)}})$ and the universal property is obtained from the diagram

\begin{align*} \xymatrix@+=1.25cm{(X,\O_X) \ar@<0.66ex>[r]^f \ar@<-0.66ex>[r]_g & (Y,\O_Y) \ar[ddr]_t \ar[dr]_{\pi} \ar[r]^{\pi} & (B,\O_{\operatorname{Qout}(B)}) \ar[d]^{\id}\\
																								&	&	 (B,\O_B) \ar[d]^{\phi} \\ & & (T,\O_T) } \end{align*}

and the characteristic property of the quotient topology.

\vspace{6.4cm}
\begin{flushright}
\textsl{\small    Felix Hoffmann\\ \href{http://felix11h.github.io}{felix11h.github.io}}
\end{flushright}

																									 
\end{document}
